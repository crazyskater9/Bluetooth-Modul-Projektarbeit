\chapter{Einleitung}
\section{Allgemeines}
Ein Audio-Bluetooth-Modul soll in einfacher Weise ein Audio-Signal von beispielsweise einem Smartphone ausgeben. Dabei ist eine hohe Kompatibilität mit viele Geräten wichtig, weil es sehr viele verschiedene Versionen von Bluetooth gibt. Da Bluetooth-Geräte meist abwärtskompatibel sind, ist es sinnvoll das Modul mit einer älteren BT-Version laufen zu lassen.

\section{Zielsetzung}
Es soll ein Print angefertigt werden auf dem sich das BT-Modul samt Versorgungsschaltung befindet. Auf diesem Print wird zusätzlich noch eine Additionsschaltung vorgesehen, um auch mit einem Klinkeneingang ein Signal zuführen zu können, falls das BT-Modul ausfällt.\\
Um eine leichtere Handhabung zu ermöglichen, muss auch ein Adapterprint für das BT-Modul angefertigt werden.

\section{Auswahl des Bluetooth-Moduls}
Wie bereits erwähnt soll das BT-Modul mit möglichst viele Geräten kompatibel sein, also mit einer älteren BT-Version laufen. Es sollte weiterhin eine möglichst einfache Bedienung für den Benutzer ermöglichen (beispielsweise Play-/Pausetaste).\\
Außerdem soll es bei geringen Kosten eine möglichst gute Verbindung, d.h. einen hohe Reichweite, erzielt werden.\\ \\
Nach ausführlicher Recherche wurde das Modul \enquote{XS3868 Revision 3} ausgewählt. Der darauf verbaute Chip \enquote{OVC3860} von \enquote{OmniVision Technologies} hat sich bereits in vielen anderen Projekten bewährt.